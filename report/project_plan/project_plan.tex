\documentclass[a4paper]{article}

\usepackage{array}
\usepackage[utf8]{inputenc}
\usepackage{graphicx}
\usepackage{amssymb}
\usepackage[bookmarks]{hyperref}
%\usepackage{multirow}
\usepackage[activate={true,nocompatibility},final,tracking=true,kerning=true,spacing=true,factor=1100,stretch=10,shrink=10]{microtype}

\begin{document}
\title{	Bachelor Project: Interactive Global Illumination \\
  \textbf{Project Plan and Description}}
\author{Jonas H. Nielsen, s121751}
\date{\parbox{\linewidth}{\centering%
  \today\endgraf\bigskip
  Supervisors\endgraf
  Niels Jørgen Christensen \hspace*{3cm} Jeppe Revall Frisvad\endgraf\medskip
  DTU Compute \endgraf
  Technical University of Denmark}}
\maketitle

\section{Project Description}
I wish to implement, document, analyse and discuss a reasonably fast screen-space method for global illumination. The base of this will be formed by the use of a deep G-buffer with minimal separation between layers as described in the 2014 paper "Fast Global Illumination Approximations on Deep G-Buffers."  While GI encompasses a multitude of methods, my primary focus will be on multi-bounce diffuse-diffuse interreflectance. While physical correctness (i.e. of energy/light transfer) will be part of solving this problem, and at the very least will be discussed in the paper, the main goal will be to produce effects that look right, without necessarily being entirely correct. That is to say, the implementation will be a compromise between physical correctness and lowering the time to draw a frame.

Specifically and in order, the goals of the project are to 1) Make a very simple framework to work and perform tests in (using GL, GLEW, GLFW, GLM and AssImp,) 2) Implement a deep G-buffers rendering method 3) Implement a diffuse-diffuse interreflectance approximation using deep G-buffers. Ideally I would reach acceptable real-time drawing times of (20ms or less) on my most powerful testing platform (1.8 TFlOPS) for acceptable quality screen-space radiosity, but I would consider 50ms and less (20fps or more) an acceptable result. If there's still development time left after reaching these goals it would be used to implement other screen-space GI methods like SSAO and dealing with some of the limitations inherent to screen-space radiosity. For instance, if a brightly coloured diffuse object escapes the viewing frustrum, it could create a disruption in quality, since the colour of the scene could change drastically. A way to fix this could be to combine the screen-space method with a global one (such as virtual point lights.) Again, this only becomes relevant if there's time to spare for development, but it would be a nice addition to the project.

\section{Project Plan}
\begin{tabular}{|l l|p{8cm}|}
\hline
\textbf{Period} & & \textbf{Task} \\
\hline

\hline
July 6\textsuperscript{th} & - July 20\textsuperscript{th} & Framework Development \\
\hline
July 20\textsuperscript{th} & - July 31\textsuperscript{st} & Implementing deep G-buffers \\
\hline
August 1\textsuperscript{st} & - August 31\textsuperscript{st} & \emph{No ECTS points} \\
\hline
September 1\textsuperscript{st} & - September 5\textsuperscript{th} & Continue work on deep G-buffer implementation \\
\hline
September 5\textsuperscript{th} & - September 12\textsuperscript{th} & Write Design and Implementation chapters on the Deep G-buffers implementation \\
\hline
September 12\textsuperscript{th} & - September 26\textsuperscript{th} & Implement screen-space radiosity method using Deep G-Buffers \\
\hline
September 27\textsuperscript{th} & - October 4\textsuperscript{th} & Write Design and Implementation chapters on the radiosity method and perform tests on performance \\
\hline
October 4\textsuperscript{th} & - October 31\textsuperscript{st} & Buffer period. If either of the two other implementations are not done, I finish them here, if they are, this period can be used for the following:
\begin{itemize}
\item SSAO
\item Occlusion testing for the radiosity method
\item Adding normal mapping to the G-Buffer generation
\item Optimisations
\item An AA method taking advantage of the deep G-Buffer
\item More in-depth testing
\end{itemize} \\
\hline
November 1\textsuperscript{st} & - December 5\textsuperscript{th} & Writing remaining part of the report including introduction, results, discussion, conclusion and abstract. If parts of design or implementation are still unfinished these are finished as well. \\
\hline
December 6th & & Delivery of project and report. \\
\hline
\end{tabular}

\end{document}